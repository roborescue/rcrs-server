\documentclass{article}

\title {Robocup Rescue Simulation - Communication Channels}
\author {Mohammad Mehdi Saboorian}
\date{February 2007\\Rev.2}

\begin{document}
\maketitle
\section{Introduction}
Till now, agents could only discriminate between unread received
messages based on senders of them. Now with introducing channels,
agents can easily decide which channel(s) they want to listen to. So
there will be more information about a message available for agents.
On the other hand, finally, kernel can force constraint on number of
messages each agent reads.

\section{Protocol changes}
Agent developers should consider these changes in order to use new
communication system. Please note that for detailed information
about rescue simulation communication protocol (including channels)
you should see {\sl "Robocup Rescue Simulation - Communication
Protocol"}.
\begin{itemize}
	\item From now on, \textbf{AK\_TELL} will be the only communication
	command. And it'll include an extra field for specifying target channel.
	\item \textbf{AK\_SAY} will be completely ignored by kernel.
	\item \textbf{KA\_HEAR} now includes a channel number, that
	specifies source of it's data.
	\item Kernel will not send \textbf{KA\_HEAR\_SAY} and 
	\textbf{KA\_HEAR\_TELL} anymore.
	\item Agents should use the new \textbf{AK\_CHANNEL} command to
	inform kernel about their preferences for reading messages.
	For instance, in a simple scenario, agent \textsl{A} selects
	channel 1 and 2. (We assume that each agent can read maximum of 4
	messages at each cycle.) Now with this command kernel will remember
	to forward maximum of 2 messages from channel 1 and 2 messages from
	channel 2 to \textsl{A}.
\end{itemize}

Those teams using \textsl{librescue} can use these new features easily.
Required changes for \textsl{Yab} are listed at appendix~\ref{yab}. Also, 
\textsl{RescueCore} will support channel based communication in near
future.

\section{Communication rules}
\begin{itemize}
	\item Previous constraint on number of messages for agents are still valid.
	\item Now all channels are accessible for each agent.
	\item Agents can read their own message with no restriction.
	\item Currently two types of channels are supported:
	\begin{enumerate}
		\item Say channel (channel 0)
		\item Radio channels
	\end{enumerate}
	\item Say channel is a replacement for AK\_SAY messages and have
	similar restrictions. (only hearable within a specified range)
\end{itemize}
 
\newpage 
\appendix

\section{Yab API}
\label{yab}
\textbf{in RCRSSProtocolSocket.java}
\begin{verbatim}
    public void akTell (int selfId, int channel, String message)
    {
        send(AK_TELL, new Object[] {
            new Integer(selfId), new Integer(channel),   
            message});
    }
    
    public void akChannel (int selfId, byte[] Channels)
    {
        send (AK_CHANNEL, new Object[] {
            new Integer (selfId), Channels});  
    }
\end{verbatim}
\textbf{in KaHear.java}
\begin{verbatim}
    public final int channelId;
    public KaHear(DataInputStream dis) throws IOException 
    {
        selfId = dis.readInt();
        senderId = dis.readInt();
        channelId = dis.readInt();
        message = RCRSSProtocol.readStringElement(dis);
    }
\end{verbatim}
\textbf{in ProtocolConstants.java} \\
\verb$    AK_CHANNEL		 = 0x90; $

\end{document}
